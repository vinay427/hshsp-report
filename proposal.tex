\documentclass{article}
\usepackage{fancyhdr,lastpage}
\setlength{\headheight}{15.2pt}
\pagestyle{fancy}
\begin{document}
\rhead{\fancyplain{}Page {\thepage} of {\pageref{LastPage}}}
\lhead{\fancyplain{}Section {\thesection}}
\title{Bioinformatics Analysis of Effect of Hydrogen Peroxide on \emph{Beta vulgaris} Seeds}
\author{Vinay Hiremath}
\date{16 July 2012}
\maketitle
\section{Introduction}
	It has recently been found through preliminary experiments that various types of seeds of the species \emph{Beta vulgaris} germinate significantly more successfully when placed in a solution of 0.3\% hydrogen peroxide (H$_{2}$O$_{2}$) as opposed to a solution of pure water (H$_{2}$O). There is a noticeable difference in both germination time (shorter) and the percentage of seeds (higher) that are germinated after four days. It is likely that this change in success rates is a direct effect of a change in the genes being expressed by these seeds. After obtaining DNA sequences of the seeds in both the control and experimental group, I will attempt to use various bioinformatic tools in conjunction to arrive at a result that clearly expresses the specific DNA sequences that are expressed only in the experimental (H$_{2}$O$_{2}$) group.
	\subsection{Hypothesis}
	As a result of the extensive procedure necessary to properly sequence the DNA of the trial groups, the limited data to be analyzed may be slightly lacking in scope. However, by comparing the genes only expressed in the experimental with annotated genomes for both \emph{Beta vulgaris} and other similar organisms, it is hopeful to maintain that the isolated expressed DNA sequences have similar listed roles in these genomes. Through this assimilation of previously gathered data, the expression of certain genes to affect the success rate of germination of the various seeds should become clear after extensive gene analysis.
	\subsection{Methods}
	I will be using a large suite of bioinformatics tools, all of which are open-sourced under the GNU Public License as freely redistributable. Annotated genome data will becollected from sources such as the Basic Local Alignment Search Tool (BLAST), Arabidopsis Information Resource (TAIR), and the Kyoto Encyclopedia of Genes and Genomes (KEGG).\\\\
	In order to first build an index for the sequence files so that they can be more easily indexed, I first needed to improve the overall quality of the data by removing poor quality data. To generate detailed quality reports, I used the fastqc toolset. After identifying which parts of my data needed trimming, I used fastx tools to complete this job. Next, I used a tool called \emph{bowtie2} as it enables strong compatibility with the assembly program I used, known as \emph{tophat}. Using this suite of programs, I was able to assemble the data I had gathered onto the RefBeet 0.9 genome, made by the Max Planck Institute for Molecular Genetics. After this, I was able to analyze the remaining assembled data sets using the cufflinks suite. Cufflinks generated indexes of each data set for both H$_{2}$O and H$_{2}$O$_{2}$. Next, cuffdiff was able to find the sections of each contig that was different between the data sets. To convert these contigs to actual sequences as represented in the RefBeet genome, I made a custom script. After these sequences were found, I submitted them to a large variety of BLAST databases as previously listed.
	\subsection{Raw Data}
	Quality before modification: \\\\ Quality after modification: \\\\ Bowtie2 Index Information: \\\\ RefBeet 0.9 Genome Information: \\\\ Tophat Assembly Information: \\\\ Cufflinks Index Information: \\\\ Cuffdiff Output Information: \\\\ Sequence List Information: \\\\ Example of BLAST Database:
	\subsection{

	\end{document}
